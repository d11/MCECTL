\documentclass[11pt]{article}
\usepackage{cite}
\usepackage{verbatim}

\begin{document}

\title{Model Checking Extended Computation Tree Logic}
\author{Daniel Horgan}
\date{Today}
\maketitle

\begin{comment}
[scratch]
Blablabla said Nobody ~\cite{Nobody06}.
\end{comment}

\begin{abstract}

TODO abstract

\end{abstract}

\section{Introduction}

Computation Tree Logic (CTL) is an effective way of modelling the behaviour of simple systems over time, but there are properties which are beyond its power to describe. More powerful logics such as CTL* (and ultimately the modal mu-calculus) have the disadvantage that their model-checking problems are intractable (EXPTIME as opposed to PTIME). A recent approach by Kreutzer et al \cite{Kreutzer10} introduces the 'Extended CTL' family of logics, in which CTL is parameterised by various classes of automata. This increases the expressive power of the logic, whilst (in some cases) preserving the tractability of model-checking. This project aims to implement a model checker for the PTIME extensions of CTL described in \cite{Kreutzer10}.

% move 1: background
%        - why is area important?
%        - background info
%        - previous research
% move 2: problem/need
% move 3: presenting project
%        - purposes,aims,objectives
%        - work carried out
%        - justification/importance
%        - outline of structure of report

\section{Background}

\subsection{Extended Computation-tree logic}
Description of the formalism...

\section{Requirements}

\section{Design}

\section{Testing}

\section{Conclusions}

\section{Acknowledgements}

I am very grateful to Stephan Kreutzer for acting as my supervisor for this project, in which capacity he ... TODO

TODO

\section{References}
\bibliography{references}{}
\bibliographystyle{plain}



\end{document}
